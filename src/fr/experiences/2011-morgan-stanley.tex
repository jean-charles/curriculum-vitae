\documentclass[../curriculum-vitae.tex]{subfiles}
\begin{document}
    \chapter{Novembre 2011/Décembre 2020 Morgan Stanley - Développeur Java Sénior/Architecte Solution (Permanent)}
        Développement d'applications d'entreprise, gestionnaire de livraison et Architecte Solution – Retenue d'impôt
        institutionnel aux États-Unis.
        \section{Morgan Stanley}
            \begin{description}
                \item[Lieu] Montréal, Canada.
                \item[Nombre d'employés] 60 000 employés.
                \item[Activité] Banque d'investissement et gestion d'actifs.
            \end{description}
            \subsection{871m}
                Développeur, chef de projet et Architecte Solution à la conception et à la mise en œuvre d'un nouveau
                système de capture de transactions et d'agrégation de flux de trésorerie pour les dérivés actions.
                \begin{itemize}
                    \item Utilisation de l'architecture de Taxman 3
                    \item Création de la documentation applicative
                    \item Tests unitaires implémentés à l'aide de JUnits et Mockito
                \end{itemize}
            \subsection{Taxman 3}
                Création de l'architecture Taxman 3 utilisant Spring Integration afin de décupler les performances
                brutes.
                \begin{itemize}
                    \item Gestion d'une équipe de 4 développeurs pour la mise en oeuvre de l'architecture
                    \item Mise en oeuvre des tests de bout en bout avec Spring test et DB Unit
                \end{itemize}
            \subsection{Taxman 2}
                Conception et mise en œuvre de la transaction fiscale financière sur les actions et les produits dérivés
                pour la France et l'Italie. Travail sur le backend avec MQ, DB2 et Spring.
                \begin{itemize}
                    \item Développement de nouvelles fonctionnalités dans l'application de calcul des taxes Taxman 2
                    \item Maintenance d'un système haut profil/haut volume en Java
                    \item Utilisation de technologies Java de pointe telles que JMS, CXF, Hibernate et Spring Framework
                    \item Coordonner l'intégration avec l'équipe de documentation des clients
                \end{itemize}
            \subsection{Customer Tax Portal}
                Dévelopement et conception d'un portail web en Flex. Modification de l'architecture pour migrer le
                portail sur ExtJS de Sencha  et de sockets Web pour transmettre la notification de mise à jour des
                données à nos utilisateurs.
                \begin{itemize}
                    \item Développement Flex et Javascript du portail
                    \item Intégration des clients suisses dans le portail avec ajout de sécurité d'accès aux données
                \end{itemize}
            \subsection{L'environnement de travail}
                \begin{itemize}
                    \item Eclipse
                    \item Git
                    \item Java
                    \item Spring
                    \item Spring Integration
                    \item Spring test
                    \item Mockito
                    \item DBUnit
                    \item Multithreading
                    \item ExtJS
                    \item MQ
                    \item JMS
                    \item DB2
                    \item Jira
                    \item Splunk
                    \item Perl
                    \item UML
                    \item MS Project
                \end{itemize}
\end{document}

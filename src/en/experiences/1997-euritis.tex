\documentclass[../curriculum-vitae.tex]{subfiles}
\begin{document}
    \chapter{1997/1999 Euritis - Développeur C \& Java (Permanent)}
        Développeur C, C++ et Java. Responsable de la maintenance de la base de donnée documentaire.
        \section{Euritis}
            \begin{description}
                \item[Lieu] Marseille, France.
                \item[Nombre d'employés] 25 employés.
                \item[Activité] Société de développement logiciel.
            \end{description}
            \subsection{Projets Européens}
                \begin{itemize}
                    \item Développement d'applets signées utilisant des cartes à puce pour signer numériquement des documents électronique.
                    \item Développement d'un proxy OCF pour piloter les lecteurs de cartes.
                \end{itemize}
            \subsection{Mise oeuvre d'un environnement de développement pour JavaCard GSM}
                \begin{itemize}
                    \item Étude et mise en oeuvre d'un environnement de développement pour applet-GSM.
                    \item Développement d'un pilote d'application JavaCard.
                \end{itemize}
            \subsection{Développement d'un serveur de base de donnée documentaire}
                \begin{itemize}
                    \item Maintenance logicielle du serveur de base de données documentaire.
                    \item Développement en Langage C sur serveur SUN et Windows NT.
                \end{itemize}
            \subsection{L'environnement de travail}
                \begin{itemize}
                    \item C
                    \item Java
                    \item OCF: Open Connectivity Foundation
                    \item Carte à puce
                    \item JavaCard
                    \item Cartes GemXpresso
                    \item PVCS
                    \item Visual Café
                    \item SUN OS
                    \item Windows NT
                    \item Microsoft Visual C++
                    \item GCC
                \end{itemize}
\end{document}
